\documentclass[letterpaper]{resume}


% This is just to get better error messages for debugging:
\setcounter{errorcontextlines}{100}

% For more compact lists.
\usepackage{paralist}

\usepackage{hyperref}

\hypersetup{
    urlcolor=cyan,           % color of external links
    colorlinks=false
}


\begin{document}

\author{Buck Shlegeris}
\email{bshlegeris@gmail.com}
\webpage{bshlgrs.github.io}
\phone{(650) 660-9155}
\maketitle

\section{Employment}

\affiliation[Software engineer]{Paypal}{Jan 2015 -- present}
\begin{compactitem}
\item Scala developer, working with Cassandra, Couchbase, Akka, etc
\end{compactitem}


\affiliation[Teaching Assistant]{App Academy}{Jan 2014 -- Jul 2014}

\begin{compactitem}
\item Developed curriculum. Wrote and presented lectures. Provided one-on-one instruction and feedback. Developed and maintained internal Rails and Backbone tools. Taught Ruby, Rails, Javascript, and Backbone.js. Interviewed and vetted applicants.

\end{compactitem}

\section{Education}

\affiliation[Bachelor of Science (Computer Science, minoring in Physics)]
            {Australian National University}{2012-2014}

\begin{compactitem}
\item Undergraduate coursework: Algorithms, operating systems, AI, algorithmic information theory and universal AI, theory of programming languages, computer architecture, linear algebra and ODEs, theory of computation
\item Director and presenter at CompCon, an inaugural Australian undergraduate CS conference; presented on algebraic behaviour of data structures
\item Completed two research projects and a variety of advanced undergraduate courses ahead of my year level.
\end{compactitem}

\section{Selected projects/conference presentations}

\affiliation[Ruining the coding interview (github.com/bshlgrs/ruining-the-coding-interview)]
            {personal project}{March 2015 -- present}
\begin{compactitem}
\item Personal project to make a new kind of optimizing compiler, which automatically selects data structures based on analysis of the usage of objects
\item Presented at Scala By The Bay 2015 (https://www.youtube.com/watch?v=oPFga7eg3Uw)
\item Written in functional Scala
\end{compactitem}

\affiliation[]
            {rPeANUt compiler [WIP] (bshlgrs.github.io/rpc/rpc)}{mostly May 2014-- July 2014}
\begin{compactitem}
\item Compiler from a subset of C including pointer arithmetic to a RISC instruction set
\item Currently partially deployed to the web as ScalaJS.
\end{compactitem}

\affiliation[]
            {Graphical Equation Manipulator, Python prototype (github.com/bshlgrs/pygem)}{2013}
\begin{compactitem}
\item Software for manipulation of equations in physics. Like Mathematica but user friendly and aimed at physics students.
\item Used Python, Sympy, Tkinter.
\item Developed software from conception to prototype to user studies with eleven users.
\item All eleven subjects thought the software let them work faster than Mathematica did.
\end{compactitem}


\section{Skills}
\textit{General:} Machine learning, deep learning, algorithms, operating systems, computer systems, C/C++/Java. Full stack web development, mostly in Rails, React, and Scala.
\textit{Languages:} Scala, Ruby, Python, Javascript, Haskell\\


\end{document}