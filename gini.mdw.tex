\input webmac
---
layout: post
title:  "Calculating the Gini coefficient"
date:   2016-09-04
---

If $$ x$$ is a sorted list of values of length $$ n$$, then the Gini
coefficient $$G$$ is given by:

$^$ G = \frac{\sum_{i=0}^n \sum_{j=0}^n \text{abs}(x_i - x_j) }{2 \cdot n^2 %
\cdot \text{mean}(x)} $^$

Where does this come from?

Here's one way of looking at it. The [mean absolute
difference](https://en.wikipedia.org/wiki/Mean_absolute_difference) of a
distribution is the expected value of drawing two items from the distribution
and returning the absolute value of their difference. Here's a formula for the
mean absolute difference $$ MD$$ of the sorted list $$ x$$ from before:

$^$MD = \frac{\sum_{i=0}^n \sum_{j=0}^n \text{abs}(x_i - x_j)}{n^2}$^$

Using this, we can write the Gini coefficient as

$^$G = \frac{MD}{2 \cdot \text{mean}(x)}$^$

This makes more sense. We divide by the mean because we want the Gini
coefficient of a distribution to be unaffected by multiplying by a constant.
($$ \frac{MD}{\text{mean}(x)}$$ is known as [relative mean absolute
distance](https://en.wikipedia.org/wiki/Mean_absolute_difference#Relative_mean_%
absolute_difference).) And then we multiply by 2 so that our Gini coefficient
varies between 0 meaning perfect equality and 1 meaning perfect inequality.

## Calculating it

We can calculate this directly. But this takes $$O(n^2)$$, which is super slow.

Alternatively, we can use a dynamic programming approach to cut the runtime
down to $$O(n)$$ plus the cost of getting the list in sorted order (which is $$
O(n \log(n))$$ if we start out with an unsorted list of incomes, but we
plausibly get our data in some other form than that).

Let's look at the sum on the numerator of the Mean Absolute Difference formula:

$^$ \sum_{i=0}^n \sum_{j=0}^n \text{abs}(x_i - x_j) $^$

Here's a table of absolute differences for items in the list $$[1, 3, 4, 5]$$.

<style>
.table {
  margin: auto;
  text-align: center;
}

.table td,th {
  padding-left: 5px;
  padding-right: 5px;
}
</style>

<table class="table">
  <tbody>
    <tr><th> </th><th>1</th><th>3</th><th>4</th><th>5</th></tr>
    <tr><th>1</th><td>0</td><td>2</td><td>3</td><td>4</td></tr>
    <tr><th>3</th><td>2</td><td>0</td><td>1</td><td>2</td> </tr>
    <tr><th>4</th><td>3</td><td>1</td><td>0</td><td>1</td></tr>
    <tr><th>5</th><td>4</td><td>2</td><td>1</td><td>0</td></tr>
  </tbody>
</table>

First thing to notice here is that the table is symmetrical around its
diagonal. If we can calculate the sum of one of those sides, we're done. Let's
choose the lower triangle.

<table class="table">
  <tbody>
    <tr><th> </th><th>1</th><th>3</th><th>4</th><th>5</th></tr>
    <tr><th>1</th><td>0</td><td> </td><td> </td><td> </td></tr>
    <tr><th>3</th><td>2</td><td>0</td><td> </td><td> </td> </tr>
    <tr><th>4</th><td>3</td><td>1</td><td>0</td><td> </td></tr>
    <tr><th>5</th><td>4</td><td>2</td><td>1</td><td>0</td></tr>
  </tbody>
</table>

Let's see how to compute the sum of that triangle efficiently.

First, let's compute a list $$ y$$ of prefix sums of $$ x$$--that is, $$ y_i$$
is the sum of the first $$ i$$ elements in $$ x$$. So $$ y_0 = 0$$, and $$ y_i
= y_{i-1} + x_{i-1}$$.

Let's see what this list looks like. Let's also look at the sum of the elements
in the $$i$$th row of the table.

<table class="table">
  <tbody>
    <tr><th>$$i$$</th><th>0</th><th>1</th><th>2</th><th>3</th></tr>
    <tr><th>$$x_i$$</th><td>1</td><td>3</td><td>4</td><td>5</td></tr>
    <tr><th>$$y_i$$</th><td>0</td><td>1</td><td>4</td><td>8</td></tr>
    <tr><th>$$\sum_{j=0}^i x_i -
x_j$$</th><td>0</td><td>2</td><td>4</td><td>7</td></tr>
  </tbody>
</table>

Can we rewrite this

$$\begin{align}
  &\sum_{j=0}^{i} \left| x_i - x_j \right|\\
  = &\sum_{j=0}^{i} \left(x_i - x_j \right) \\
  = &i\cdot x_i - \left( \sum_{j=0}^i x_j \right)
  \end{align}$$

Or, taking advantage of $$y$$:

$^$i\cdot x_i - \left( \sum_{j=0}^i x_j \right) =i \cdot x_i - y_i$^$

Feel free to verify that this is correct:

<table class="table">
  <tbody>
    <tr><th>$$i$$</th><th>0</th><th>1</th><th>2</th><th>3</th></tr>
    <tr><th>$$x_i$$</th><td>1</td><td>3</td><td>4</td><td>5</td></tr>
    <tr><th>$$y_i$$</th><td>0</td><td>1</td><td>4</td><td>8</td></tr>
    <tr><th>$$\sum_{j=0}^i x_i -
x_j$$</th><td>0</td><td>2</td><td>4</td><td>7</td></tr>
    <tr><th>$$i \cdot x_i -
y_i$$</th><td>0</td><td>2</td><td>4</td><td>7</td></tr>
  </tbody>
</table>

Now we can write a super simple for loop to calculate our Gini coefficient:

<<>>=
print("hello world")

\M1.
\fi


\inx
\fin
\con
